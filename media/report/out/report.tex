\documentclass[a4paper, 12pt]{ctexart}
% generated by Madoko, version 1.1.3
%mdk-data-line={1}


\usepackage[heading-base={2},section-num={False},bib-label={hide},fontspec={True}]{madoko2}


\begin{document}



%mdk-data-line={6}
\mdxtitleblockstart{}
%mdk-data-line={6}
\mdxtitle{\mdline{6}数字媒体技术}%mdk
\mdtitleauthorrunning{}{}\mdxtitleblockend%mdk

%mdk-data-line={8}
\mdhr{}%mdk

%mdk-data-line={9}
\section{\mdline{9}1.\hspace*{0.5em}\mdline{9}PCA}\label{sec-pca}%mdk%mdk

%mdk-data-line={10}
\subsection{\mdline{10}1.1.\hspace*{0.5em}\mdline{10}背景:}\label{section}%mdk%mdk

%mdk-data-line={11}
\noindent\mdline{11}在多元统计分析中,主成分分析(英语:Principal components analysis,PCA)是一种分析、简化数据集的技术。主成分分析经常用于减少数据集的维数,同时保持数据集中的对方差贡献最大的特征。这是通过保留低阶主成分,忽略高阶主成分做到的。这样低阶成分往往能够保留住数据的最重要方面。但是,这也不是一定的,要视具体应用而定。由于主成分分析依赖所给数据,所以数据的准确性对分析结果影响很大。%mdk

%mdk-data-line={13}
\subsection{\mdline{13}1.2.\hspace*{0.5em}\mdline{13}数学定义:}\label{section}%mdk%mdk

%mdk-data-line={14}
\noindent\mdline{14}PCA的数学定义是:一个正交化线性变换,把数据变换到一个新的坐标系统中,使得这一数据的任何投影的第一大方差在第一个坐标(称为第一主成分)上,第二大方差在第二个坐标(第二主成分)上,依次类推\mdline{14}[4]\mdline{14}。
定义一个n × m的矩阵, XT为去平均值(以平均值为中心移动至原点)的数据,其行为数据样本,列为数据类别(注意,这里定义的是XT 而不是X)。则X的奇异值分解为X = WΣVT,其中m × m矩阵W是XXT的本征矢量矩阵, Σ是m × n的非负矩形对角矩阵,V是n × n的XTX的本征矢量矩阵。据此,%mdk

%mdk-data-line={18}
\mdline{18}\mdline{24}
\mdline{25}\mdline{25}
当 m \mdline{26}\textless{}\mdline{26} n − 1时,V 在通常情况下不是唯一定义的,而Y 则是唯一定义的。W 是一个正交矩阵,YT是XT的转置,且YT的第一列由第一主成分组成,第二列由第二主成分组成,依此类推。
为了得到一种降低数据维度的有效办法,我们可以利用WL把 X 映射到一个只应用前面L个向量的低维空间中去:%mdk


%mdk-data-line={88}
\noindent\mdline{88}X 的单向量矩阵W相当于协方差矩阵的本征矢量 C = X XT,\mdline{88}\mdline{88}
\mdline{89}\mdline{89}%mdk








%mdk-data-line={367}
\subsection{\mdline{367}1.3.\hspace*{0.5em}\mdline{367}算法实现一般过程:}\label{section}%mdk%mdk

%mdk-data-line={368}
\begin{enumerate}[noitemsep,topsep=\mdcompacttopsep]%mdk

%mdk-data-line={368}
\item\mdline{368}求平均值以及做normalization%mdk

%mdk-data-line={369}
\item\mdline{369}求协方差矩阵(Covariance Matrix)%mdk

%mdk-data-line={370}
\item\mdline{370}求协方差矩阵的特征根和特征向量%mdk

%mdk-data-line={371}
\item\mdline{371}选择主要成分(信息量)%mdk

%mdk-data-line={372}
\item\mdline{372}转化得到降维的数据%mdk
%mdk
\end{enumerate}%mdk

%mdk-data-line={374}
\subsection{\mdline{374}1.4.\hspace*{0.5em}\mdline{374}实验结果}\label{section}%mdk%mdk

%mdk-data-line={375}
\noindent\mdline{375}\includegraphics[keepaspectratio=true,width=\dimmin{}{\dimwidth{0.90}}]{images/PCA}{}\mdline{375}%mdk

%mdk-data-line={379}
\section{\mdline{379}2.\hspace*{0.5em}\mdline{379}JPEG压缩原理与DCT离散余弦变换}\label{sec-jpegdct}%mdk%mdk

%mdk-data-line={380}
\subsection{\mdline{380}2.1.\hspace*{0.5em}\mdline{380}背景:}\label{section}%mdk%mdk

%mdk-data-line={381}
\noindent\mdline{381}DCT变换的全称是离散余弦变换(Discrete Cosine Transform),主要用于将数据或图像的压缩,能够将空域的信号转换到频域上,具有良好的去相关性的性能。DCT变换本身是无损的,但是在图像编码等领域给接下来的量化、哈弗曼编码等创造了很好的条件,同时,由于DCT变换时对称的,所以,我们可以在量化编码后利用DCT反变换,在接收端恢复原始的图像信息。DCT变换在当前的图像分析已经压缩领域有着极为广大的用途,我们常见的JPEG静态图像编码以及MJPEG、MPEG动态编码等标准中都使用了DCT变换。%mdk

%mdk-data-line={383}
\mdline{383}\mdline{383}
\mdline{384} \mdline{384}%mdk

%mdk-data-line={386}
\subsection{\mdline{386}2.2.\hspace*{0.5em}\mdline{386}数学定义:}\label{section}%mdk%mdk

%mdk-data-line={388}
\subsubsection{\mdline{388}2.2.1.\hspace*{0.5em}\mdline{388}一维DCT变换}\label{sec-dct}%mdk%mdk

%mdk-data-line={389}
\noindent\mdline{389}一维DCT变换时二维DCT变换的基础,所以我们先来讨论下一维DCT变换。一维DCT变换共有8种形式,其中最常用的是第二种形式,
由于其运算简单、适用范围广。我们在这里只讨论这种形式,其表达式如下:%mdk

%mdk-data-line={393}
\mdline{393}\mdline{393}%mdk

%mdk-data-line={396}
\mdline{396}其中,f(i)为原始的信号,F(u)是DCT变换后的系数,N为原始信号的点可以认为是一个补偿系数,可以使DCT变换矩阵为正交矩阵。%mdk

%mdk-data-line={398}
\subsubsection{\mdline{398}2.2.2.\hspace*{0.5em}\mdline{398}二维DCT变换}\label{sec-dct}%mdk%mdk

%mdk-data-line={399}
\noindent\mdline{399}二维DCT变换其实是在一维DCT变换的基础上在做了一次DCT变换,其公式如下:%mdk

%mdk-data-line={401}
\mdline{401}\mdline{401}
\mdline{402}\mdline{402}
       由公式我们可以看出,上面只讨论了二维图像数据为方阵的情况,在实际应用中,如果不是方阵的数据一般都是补齐之后再做变换的,重构之后可以去掉补齐的部分,得到原始的图像信息,这个尝试一下,应该比较容易理解。%mdk

%mdk-data-line={405}
\mdline{405}另外,由于DCT变换高度的对称性,在使用Matlab进行相关的运算时,我们可以使用更简单的矩阵处理方式:%mdk

%mdk-data-line={407}
\mdline{407}\mdline{407}%mdk

%mdk-data-line={410}
\subsection{\mdline{410}2.3.\hspace*{0.5em}\mdline{410}JPEG压缩流程:}\label{sec-jpeg}%mdk%mdk

%mdk-data-line={411}
\begin{enumerate}[noitemsep,topsep=\mdcompacttopsep]%mdk

%mdk-data-line={411}
\item\mdline{411}以8x8的图象块为基本单位进行编码%mdk

%mdk-data-line={412}
\item\mdline{412}将RGB转换为亮度-色调-饱和度系统(YUV),并重新采样%mdk

%mdk-data-line={413}
\item\mdline{413}FDCT%mdk

%mdk-data-line={414}
\item\mdline{414}量化%mdk

%mdk-data-line={415}
\item\mdline{415}编码%mdk

%mdk-data-line={416}
\item\mdline{416}解码%mdk

%mdk-data-line={417}
\item\mdline{417}反量化%mdk

%mdk-data-line={418}
\item\mdline{418}IDCT%mdk

%mdk-data-line={419}
\item\mdline{419}图像拼接%mdk
%mdk
\end{enumerate}%mdk

%mdk-data-line={421}
\subsection{\mdline{421}2.4.\hspace*{0.5em}\mdline{421}实验结果:}\label{section}%mdk%mdk

%mdk-data-line={422}
\noindent\mdline{422}\textbf{DCT:}\mdline{422}\mdline{422}
\mdline{423}\includegraphics[keepaspectratio=true,width=\dimmin{}{\dimwidth{0.90}}]{images/dct}{}\mdline{423}%mdk

%mdk-data-line={425}
\noindent\mdline{425}\textbf{IDCT:}\mdline{425}\mdline{425}
\mdline{426}\includegraphics[keepaspectratio=true,width=\dimmin{}{\dimwidth{0.90}}]{images/idct}{}\mdline{426}%mdk

%mdk-data-line={430}
\noindent\mdline{430}\textbf{压缩率:10}\mdline{430}\mdline{430}
\mdline{431}\mdline{431}%mdk

%mdk-data-line={434}
\mdline{434}\mdline{434}
\mdline{435}\textbf{压缩率:50}\mdline{435}\mdline{435}
\mdline{436}\mdline{436}
\mdline{437}\mdline{437}
JPEG压缩比例,就是通过控制量化的多少来控制。比如,上面的量化矩阵Q,如果我把矩阵的每个数都double一下,那是不是会出现更多的0?!说不定都只有G(0, 0)非0,其他都是0,如果这样,那编码时就可以更省空间啦,N个0只要一个游程编码搞定,数据量超小。但也意味着,恢复时,会带来更多的误差,图像质量也会变差了。%mdk%mdk


\end{document}
